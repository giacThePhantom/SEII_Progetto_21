\documentclass[oneside]{book}
\usepackage[a4paper, total={6in, 8in}]{geometry}
\usepackage[italian]{babel}
\usepackage[utf8]{inputenc}
\usepackage[T1]{fontenc}
\usepackage{amsmath}
\usepackage{listings}
\usepackage{hyperref}
\usepackage{siunitx}
\usepackage{fancyhdr}
\pagestyle{fancy}
\usepackage{textcomp}
\usepackage{makecell}
\usepackage[font=small,labelfont=bf]{caption} 
\usepackage{pdfpages}
\usepackage{multicol}
\usepackage{multirow}
\usepackage[ruled,vlined]{algorithm2e}
\usepackage{mhchem}
\pagestyle{fancy}
\fancyhf{}
\lhead{\rightmark}
\cfoot{\leftmark}
\rfoot{\thepage}

\setcounter{secnumdepth}{5}

\lstset{
    frame=tb, % draw a frame at the top and bottom of the code block
    tabsize=4, % tab space width
    showstringspaces=false, % don't mark spaces in strings
    numbers=none, % display line numbers on the left
    commentstyle=\color{green}, % comment color
    keywordstyle=\color{red}, % keyword color
    stringstyle=\color{blue}, % string color
    breaklines=true,
    postbreak=\mbox{\textcolor{green}{$\hookrightarrow$}\space}
}

\renewcommand{\lstlistingname}{}% Listing -> Algorithm
\renewcommand{\lstlistlistingname}{Algoritmi}% List of Listings -> List of Algorithms
\author{
  Giacomo Fantoni \\
  \small Telegram: \href{https://t.me/GiacomoFantoni}{@GiacomoFantoni} \\[3pt]
  Github: \href{https://github.com/giacThePhantom/AlgoritmiStruttureDati}{https://github.com/giacThePhantom/AlgoritmiStruttureDati}}


\renewcommand*{\listalgorithmcfname}{}
\renewcommand*{\algorithmcfname}{}
\renewcommand*{\algorithmautorefname}{}
\renewcommand{\thealgocf}{}


\title{\Huge \textbf{About}}

\author{
  Giacomo Fantoni \\
  \small Telegram: \href{https://t.me/GiacomoFantoni}{@GiacomoFantoni} \\[3pt]
  \small Github: \href{https://github.com/giacThePhantom/BioMolCellula2}{https://github.com/giacThePhantom/BioMolCellula2}}
\begin{document}
\maketitle
\tableofcontents

\chapter{Genome}
A genome sequence is the complete list of nucleotides that make up all the chromosomes of an individual or a species.
Within a species the vast majority of nucleotides are identical between individuals, but sequencing multiple individuals is necessary to understand the genetic diversity.
Eukaryotic genomes are composed of one or more linear DNA chromosomes.
The number of chromosomes varies widely from each species.
Most of the species contains a number of autosomal chromosome and a pair of sexual chromosomes.
A typical human cell has two copies of each of 22 autosomes, one inherited from each parent and two
sex chromosomes, making it a diploid.\\
\href{https://en.wikipedia.org/wiki/Genome}{Wikipedia page}
\section{Nucleotides}
Nucleotides are organic molecules consisting of a nucleoside and a fosfate.
They serve as monomeric units of the nucleic acid polymers deoxuribonucleic acid or DNA.\\
\href{https://en.wikipedia.org/wiki/Nucleotide}{Wikipedia page}
\section{DNA molecules}
A DNA molecule (deoxyribonucleic acid) is a molecule composed of two polynucleotide chains that coils around each other to form a double helix carrying genetic instructions for the development, functioning and growth of all known organisms.\\
\href{https://en.wikipedia.org/wiki/DNA}{Wikipedia page}
\section{Chromosomes}
A chromosome is a long DNA molecule with part or all of the genetic material of an organism.
Most eukaryotic chromosomes include packaging protein called histones which bind and condense the DNA molecule to maintain its integrity and keep it inside the nucleus.
These chromosomes display a complex three-dimensional structure, which plays a significant role in transcriptional regulation.
They are clearly visible during the metaphase of cell division, where they are all aligned in the centre of the cell in their condensed form.
Before this phase each chromosome is duplicated and both copies are joined by a centromere.
The centromere is a special region in a chromosome with no genetic information: it has a structural role and it is the place where special proteins bind and segregate the copies in each of the two children cells.\\
\href{https://en.wikipedia.org/wiki/Chromosome}{Wikipedia page}
\chapter{Genes}
A gene is a sequence of nucleotides that encodes the synthesis of a gene product, either RNA or protein.
During gene expression the DNA is first copied into RNA that can be directly functional or the intermediate template for a protein.
All steps of genetic expressions are highly regulated.
RNAs and proteins are the cell's main actors: they give it structure and perform all the operation needed for its survival, growth and differentiation.\\
\href{https://en.wikipedia.org/wiki/Gene}{Wikipedia page}
\chapter{Evolution}
\href{https://en.wikipedia.org/wiki/Evolution}{Wikipedia page}
Evolution is the change in heritable characteristics of biological populations over successive generations.
These characteristics are the expressions of genes passed from parent to offspring during evolution.
These changes tend to exist within any given population as a result of genetic variation.
Evolution occurs when ecolutionary processes such as natural selection and genetic drift act on this variation, resultin in certain characteristics becoming more common or rare within a population.
This process has given rise to biodiversity at every level of biological organisation, including the levels of species, individual organismos and molecules.
\section{Genetic variability}
Genetic variability is either the presence of, or the generation of, genetic differences.
It is defined as the formation of individuals differing in genotype or the presence of genotypically different individuals, in contrast to environmentally induced differences which cause only temporary, non heritable changes of the phenotype.\\
\href{https://en.wikipedia.org/wiki/Genetic_variability}{Wikipedia page}
\subsection{Phenotype}
Phenotype is the term used in genetics for the composite observable characteristics or traits of an organism.
It covers the organism's morphology or physical form and structure, its developmental processes, its biochemical and physiological properties, its behaviour and the products of behaviour.
It is the result of the expression of an organism genetic code or genotype and the influence of environmental factors.\\
\href{https://en.wikipedia.org/wiki/Phenotype}{Wikipedia page}
\subsection{Genotype}
A genotype is an organism's set of heritable genes that can be passed down from parents to offspring.
The genes take part in determining the characteristics that as observable in an organism, or its phenotype.\\
\href{https://en.wikipedia.org/wiki/Genotype}{Wikipedia page}
\subsection{Causes for genetic variability}
\begin{itemize}
\item Homologous recombination
Homologous recombination is a significant source of variability: during meiosis in sexual organism, two homologous chromosomes cross over one another and exchange genetic material.
The chromosome then split apart and are ready to contribute to forming an offspring.
It's random.\\
	\href{https://en.wikipedia.org/wiki/Homologous_recombination}{Wikipedia page}
\item Immigration, emigration and traslocation.
\item Polyploidy: having more than two homologous chromosomes allows for even more recombination during meiosis allowing for even more genetic variability in one's offspring.
\item Genetic mutations.
\end{itemize}
\subsection{Genetic mutations}
A genetic mutation is an alteration in the nucleotide sequence of the genome in an organism.
Mutations result from errors during DNA replication, mitosis or meiosis or other types of damage to DNA.
They play a central part in evolution.\\
\href{https://en.wikipedia.org/wiki/Mutation}{Wikipedia page}
\subsubsection{Mutation rate}
Mutation rates vary substantially across species.
In humans the mutation rate is about 50-09 de novo mutation per genome per generation: each human accumulates about 50-90 mutations that are not present in his or her parents.
\subsubsection{Sequence homology}
Sequence homology is the biological homology between DNA, defined in terms of shared ancestry in the evolutionary history of live.
Two segments of DNA can have shared ancestry because a speciation event, a duplication event or a horizontal gene transfer.\\
\href{https://en.wikipedia.org/wiki/Sequence_homology}{Wikipedia page}
\paragraph{Orthologous}
Orthologous sequences are inferred to be descended from the same ancestral sequence separated by a speciation event: when a species diverges in two separate species the copies of a single gene in the two resulting species are said to be orthologous.
Orthologous genes are genes in different species that originated by vertical descent from a single gene of the last common ancestor.
\paragraph{Paralogous}
Paralogous genes are genes related via duplication events in the last common ancestor of the species being compared.
They result from the mutation of duplicated genes during separate speciation events.
When descendants from the last common ancestor share mutated homologs of the original duplicated genes then those genes are considered paralogs.

\chapter{Sources}
\begin{itemize}
	\item Genes list: \href{https://www.ensembl.org/biomart/martview}{Ensembl - biomart - martview}
	\item Genes data: \href{https://rest.ensembl.org}{Ensembl rest API}
\end{itemize}
\end{document}
